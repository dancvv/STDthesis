% !TeX root = ../Template.tex
% [绪论]
% 此处为本LaTeX模板的简介
\chapter{绪论}

大家好,这是铁大论文\LaTeX{}模板(\CTeX{}-Based)---\STDThesis{}。

\STDThesis{}为铁大本科生学位论文模板,适用于本科论文写作。本\LaTeX{}模板参考了教务处下发的本科毕业论文格式,本模板基本覆盖了论文内容和格式方面的要求。Mac系统下请采用宏包并使用XeLaTeX编译。文献著录BibTeX样式采用HaixingHu开源的2005版参考文献著录BibTeX样式\href{https://github.com/Haixing-Hu/GBT7714-2005-BibTeX-Style}{GBT~7714-2005}及Zeping Lee开源的2015版参考文献著录BibTeX样式\href{https://github.com/zepinglee/gbt7714-bibtex-style}{GBT7714-2015},在此感谢两位的开源分享。请自行选用:\\
\verb|\Bib{GBT7714-2005}{yourRefFile}|或\\
\verb|\Bib{GBT7714-2015}{yourRefFile}|。

模板能够顺利成型不得不感谢LaTexStduio网站中的大量模板,站在这些前人的基础上,我才能整理出这个模板。闭门造车真的是太难了,LaTex的模板的确很多,但是没有一双会发现的眼睛,真的很难受。高德纳教授开发这个排版系统的时候就没考虑东亚语系的兼容,导致了中文用户的稀缺。模板虽多,但大多是英文且无注释的,鲜有一个好的中文注释模板。而本模板主要参考的北航硕博\LaTeX 论文模板就相当不错。

本模板已上传GitHub\footnote{\href{https://github.com/dancvv/stdthesis}{https://github.com/dancvv/stdthesis}}。
意见及问题反馈请联系:\\
\indent E-mail:uquantum@hotmail.com

\indent GitHub: \href{https://github.com/dancvv/stdthesis}{https://github.com/dancvv/stdthesis}
%%============================
\section{概述}
学位论文是标明作者从事科学研究取得的创造性成果和创新见解,并以此为内容撰写的、作为申请学位时评审用的学位论文。

硕士学位论文应该表明作者在本门学科上掌握了坚实的基础理论和系统的专门知识,对所研究的课题有新的见解,并具有从事科学研究工作或独立担任专门技术工作能力。

%%============================
\section{内容要求}
论文应立论正确、推理严谨、说明透彻、数据可靠。

论文应结构合理、层次分明、叙述准确、文字简练、文图规范。对于涉及作者创新性工作和研究特点的内容应重点论述,做到数据或实例丰富、分析全面深入。文中引用的文献资料必须表明来源,使用的计量单位、绘图规范应符合国家标准。

论文内容包括:选题的背景、依据及意义;文献及相关研究综述、研究及设计方案、实验方法、装置和实验结果;理论的证明、分析和结论;重要的计算、数据、图表、曲线及相关分析;必要的附录、相关的参考文献目录等,如表\ref{tab:papercomponents}。

\centerline{-----------$\downarrow$-----------Space Check-----------$\downarrow$-----------}
\begin{table}[h]
  \caption{学位论文组成}
  \label{tab:papercomponents}
  \centering
  \begin{tabular}{cp{16\ccwd}p{4cm}}
    \toprule
    {\bfseries 装订顺序} & \multicolumn{1}{c} {\bfseries 内容} & \multicolumn{1}{c} {\bfseries 说明}  \\
    \midrule
    1 & 封面(中、英文)& \\
    2 & 题名页          & \\
    3 & 中文摘要        & \\
    4 & 英文摘要        & \\
    5 & 目录            & \\
    6 & 图表清单及主要符号表  & 根据具体情况可省略 \\
    7 & 主体部分        & \\
    8 & 参考文献        & \\
    9& 附录            & \\
    10& 致谢            & \\
    \bottomrule
  \end{tabular}
\end{table}
\centerline{-----------$\uparrow$-----------Space Check-----------$\uparrow$-----------}

%%----------------------
\subsection{封面}

{\bfseries \uline{XX}届}:应准确填写培养的学院或独立系的全称和毕业届数。

{\bfseries 专业}:一级学科名称。

{\bfseries 姓名}:中文名,指导手册无英文封面要求,故未设定。

{\bfseries 学号}:在校学号。

{\bfseries 指导教师}:所选毕设的指导教师。




%%----------------------
\subsection{摘要}

中文摘要包括“摘要”字样,摘要正文及关键词。对于中英文摘要,都必须在摘要的最下方另起一行。

摘要是学位论文内容的简短陈述,应体现论文工作的核心思想。论文摘要应力求语言精炼准确。博士学位论文的中文摘要一般约800$\sim$1200字;硕士学位论文的中文摘要一般约500字。摘要内容应涉及本项科研工作的目的和意义、研究思想和方法、研究成果和结论。博士学位论文必须突出论文的创造性成果,硕士学位论文必须突出论文的新见解。

关键字是为用户查找文献,从文中选取出来揭示全文主体内容的一组词语或术语,应尽量采用词表中的规范词(参考相应的技术术语标准)。关键词一般3$\sim$5个,按词条的外延层次排列(外延大的排在前面)。关键词之间用逗号分开,最后一个关键词后不打标点符号。

为了国际交流的需要,论文必须有英文摘要。英文摘要的内容及关键词应与中文摘要及关键词一致,要符合英语语法,语句通顺,文字流畅。英文和汉语拼音一律为Times New Roman体,字号与中文摘要相同。

%%----------------------
\subsection{目录}

目录按章、节、条和标题编写,一般为二级或三级,目录中应包括绪论(或引言)、论文主体章节、结论、附录、参考文献、附录、攻读学位期间取得的成果等。

%%----------------------
\subsection{图表清单及主要符号表}
视情况而定,如若需要则将代码中的off改为on

如果论文中图表较多,可以分别列出清单置于目录之后。图的清单应有序号、图题和页码,表的清单应有序号、标题和页码。
全文中常用的符号、标志、缩略词、首字母缩写、计量单位、名词、术语等的注释说明,如需汇集,可集中在图和表清单后的主要符号表中列出,符号表排列顺序按英文及其相关文字顺序排出。

%%----------------------
\subsection{主体部分}

一般应包括:绪论(或引言)、正文、结论等部分。

每章应另起一页。章节标题不得使用标点符号,尽量不采用英文缩写词,对必须采用者,应使用本行业的通用缩写词。
三级标题的层次对理工类建议按章(如“第一章”)、节(如“1.1”)、条(如“1.1.1”)的格式编写;对社科、文学类建议按章(如“一、”)、节(如“(一)”)、条(如“1、”)的格式编写,各章题序的阿拉伯数字用Times New Roman字体。

本科毕业论文一般为1$\sim$2万字。

%%----------------------
\subsection{参考文献}

学术研究应精确、有据、坦诚、创新、积累。而其中精确、有据和积累需要建立在正确对待前人学术成果的基础上。凡有直接引用他人成果之处,均应加标注说明列于参考文献中,以避免论文抄袭现象的发生。

研究生论文参考文献著录及标引按照国家标准《文后参考文献著录规则》(GB774)和中国博硕士学位论文编写与交换格式。

如果你足够仔细,会发现参考文献有两份还都是相同的,这是为了准确地生成参考文献。在论文写作完成之后,你可以注释掉Template.tex中的\verb|\bibliography{reference}|代码,一切问题均可迎刃而解。但切不可动\verb|\Bib{GBT7714-2015}{reference}|这一条代码,动了它就是动摇参考文献之稷。



%%----------------------
\subsection{附录}

附录作为论文主体的补充项目,并不是必需的。

%%----------------------
\subsection{致谢}
致谢中主要感谢指导教师在和学术方面对论文的完成有直接贡献及重要帮助的团体和人士,以及感谢给予转载和引用权的资料、图片、文献、研究思想和设想的所有者。致谢中还可以感谢提供研究经费及实验装置的基金会或企业等单位和人士。致谢辞应谦虚诚恳,实事求是,切记浮夸与庸俗之词。

%%----------------------
\subsection{作者简介}

博士学位论文应该提供作者简介,主要包括:姓名、性别、出生年月日、民族、出生的;简要学历、工作经历(职务);以及攻读博士学位期间获得的其他奖项(除攻读学位期间取得的研究成果之外)。

但是本科生不需要,忽略即可。
